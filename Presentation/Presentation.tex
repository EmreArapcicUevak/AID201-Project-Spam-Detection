%! Author = emrearapcic-uevak
%! Date = 18. 12. 2025.

% Preamble
\documentclass{beamer}

\usetheme{Madrid} % Clean theme; alternatives: "metropolis", "Boadilla", "CambridgeUS"
\usecolortheme{seahorse}

% Packages
\usepackage{amsmath}
\usepackage[utf8]{inputenc}
\usepackage{graphicx}
\usepackage{hyperref}
\usepackage{booktabs}
\usepackage{qrcode}

%Information to be included in the title page:
\title{Spam Detection}
\author[Emre]{Emre Arapcic-Uevak}
\institute[IUS]{
  International University of Sarajevo \\
  Faculty of Engineering and Natural Sciences
}
\date{\today}

\AtBeginSubsection[] {
  \begin{frame}{Table of Contents}
    \tableofcontents[currentsection,currentsubsection]
  \end{frame}
}

% Document
\begin{document}
    \frame{\titlepage}

    \begin{frame}{What will we talk about?}
        \tableofcontents[pausesections]
    \end{frame}

    \section{Introduction}
    \subsection{What is the problem?}
    \begin{frame}{Problem}
        \begin{itemize}
            \item SMS Spam Detection
            \item<2-> SMS Messages are not filtered
            \item<3-> SMS Messages can be sent at a much higher volume then Emails for a low price
        \end{itemize}
    \end{frame}

    \subsection{What did we make?}
    \begin{frame}{Question}
        \begin{block}{Question}
            How would you tell if a message was \emph{Spam} or \emph{Ham}
        \end{block}

        \begin{block}<2->{Question}
            How would you tell a \alert{computer} to distinguish if a message was \emph{Spam} or \emph{Ham}
        \end{block}
    \end{frame}

    \begin{frame}{Answer}
        \pause
        It is difficult\pause... \textbf{BUT} \pause when it gets hard for
        us to define the problem we can always make the computer \alert{learn} \\
        
        \pause

        \vspace*{1cm}
        \begin{definition}[Machine Learning]
            A computer program is said to \emph{learn} from experience $E$ with respect to
            a class of tasks $T$ and a performance measure $P$ if its performance at tasks
            in $T$, as measured by $P$, improves with experience $E$.
        \end{definition}
    \end{frame}

    \section{Data set}
    \subsection{Who collected the data set?}
    \subsection{What was the purpose of the data collection?}
    \subsection{When was the data collected?}
    \subsection{Feature Analysis}
    
    \section{Model}
    \subsection{What model did we choose?}
    \subsection{Performance Analysis}
\end{document}