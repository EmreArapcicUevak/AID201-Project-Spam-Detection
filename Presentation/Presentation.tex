%! Author = emrearapcic-uevak
%! Date = 18. 12. 2025.

% Preamble
\documentclass{beamer}

\usetheme{Madrid} % Clean theme; alternatives: "metropolis", "Boadilla", "CambridgeUS"
\usecolortheme{seahorse}

% Packages
\usepackage{amsmath}
\usepackage[utf8]{inputenc}
\usepackage{graphicx}
\usepackage{hyperref}
\usepackage{booktabs}
\usepackage{qrcode}

%Information to be included in the title page:
\title{Spam Detection}
\author[Emre]{Emre Arapcic-Uevak}
\institute[IUS]{
  International University of Sarajevo \\
  Faculty of Engineering and Natural Sciences
}
\date{\today}

\AtBeginSubsection[] {
  \begin{frame}{Table of Contents}
    \tableofcontents[currentsection,currentsubsection]
  \end{frame}
}

% Document
\begin{document}
    \frame{\titlepage}

    \begin{frame}{What will we talk about?}
        \tableofcontents[pausesections]
    \end{frame}

    \section{Introduction}
    \subsection{What is the problem?}
    \begin{frame}{Problem}
        \begin{itemize}
            \item SMS Spam Detection
            \item<2-> SMS Messages are not filtered
            \item<3-> SMS Messages can be sent at a much higher volume then Emails for a low price
        \end{itemize}
    \end{frame}

    \subsection{What did we make?}
    \begin{frame}{Question}
        \begin{block}{Question}
            How would you tell if a message was \emph{Spam} or \emph{Ham}
        \end{block}

        \begin{block}<2->{Question}
            How would you tell a \alert{computer} to distinguish if a message was \emph{Spam} or \emph{Ham}
        \end{block}
    \end{frame}

    \begin{frame}{Answer}
        \pause
        It is difficult\pause\ldots \textbf{BUT} \pause when it gets hard for
        us to define the problem we can always make the computer \alert{learn} \\
        
        \pause

        \vspace*{1cm}
        \begin{definition}[Machine Learning]
            A computer program is said to \emph{learn} from experience $E$ with respect to
            a class of tasks $T$ and a performance measure $P$ if its performance at tasks
            in $T$, as measured by $P$, improves with experience $E$.
        \end{definition}
    \end{frame}

    \section{Data set}
    \subsection{Who collected the data set?}
    \begin{frame}{Collectors}
        The SMS Spam Collection was compiled and donated by researchers working on SMS spam filtering.
        It was created by \textbf{Tiago A. Almeida}, \textbf{José María Gómez Hidalgo}, and collaborators from academic
        sources. \\ \vspace{1cm}

        \pause

        The corpus itself combines messages gathered from several public SMS sources, including:
        \begin{itemize}
            \item manually extracted spam from the \textit{Grumbletext} website,
            \item SMS ham messages from the \textit{NUS SMS Corpus} (collected from volunteers),
            \item and ham/spam messages from other publicly released collections.
        \end{itemize}
    \end{frame}
    \subsection{What was the purpose of the data collection?}
    \begin{frame}{Purpose behind the data collection}
        The primary purpose of collecting this dataset was to support
        \textbf{mobile phone spam research} \pause specifically to evaluate and develop spam filtering and classification methods.

        \vspace{0.5cm} \pause
        It provides a labeled set of SMS messages (ham vs. spam) for machine learning tasks such as classification and clustering.
    \end{frame}

    \subsection{When was the data collected?}
    \begin{frame}{When was the data collected?}
        The dataset was \textbf{donated to the UCI repository on June 21, 2012}. \vspace{0.3cm} \pause \\
        The component messages themselves come from various sources gathered prior to that donation, but \textit{as a public dataset} it has been available since \textbf{2012} for research and benchmarking.
    \end{frame}

    \subsection{Feature Analysis}
    \begin{frame}{How does the dataset look?}
        \input{Tables/Spam_Before_Processing}
    \end{frame}
    
    \begin{frame}{Visualization of missing data}
        \centering
        \only<1>{
            \includegraphics[width = \textwidth, clip={0 0 0 0}]{Images/Missing_Values_Matrix}
        }

        \only<2>{
            \includegraphics[width = \textwidth]{Images/Missing_Values_HeatMap}
        }
    \end{frame}

    \begin{frame}{Taking a closer look at the unnamed columns}
        \only<1>{
            \includegraphics[height = 0.85\textheight]{Images/First_Character_Count}
        }

        \only<2>{
            \input{Tables/Mistake_Values_Example_Table}
        }
    \end{frame}
    
    \begin{frame}{After fixing up all of the mistakes}
        \input{Tables/Spam_After_Processing}
    \end{frame}

    \begin{frame}{Distribution of target class}
        \centering
        \only<1> {
            \includegraphics[width = \textwidth]{Images/Spam_Count}
        }

        \only<2> {
            \includegraphics[height = 0.85\textheight]{Images/Spam_Class_Distribution}
        }
    \end{frame}

    \section{Model}
    \subsection{What model did we choose?}
        \begin{frame}{Naive Bayes}
            In Machine learning, Naive Bayes Classifiers are a family of simple ``probabilistic classifiers`` based on applying
            Bayes' theorem with strong independence assumption between the features.
            \vspace{1mm}

            \begin{equation*}
                P(A | B) = \frac{P(B | A) \cdot P(A)}{P(B)}
            \end{equation*}


        \end{frame}
    \subsection{Performance Analysis}
\end{document}